\documentclass[osajnl,twocolumn,showpacs,superscriptaddress,10pt,floatfix]{revtex4-1} %% use 11pt for Applied Optics
\usepackage{amsmath,amssymb,graphicx,float}
\usepackage{adjustbox}
\usepackage[utf8]{inputenc}
\graphicspath{ {images/} }

\begin{document}

\title{}

\author{Ulises Jeremias Cornejo Fandos}
\affiliation{Licenciatura en Informatica - 13566/7, Facultad de Informatica, UNLP}

\begin{abstract}

\end{abstract}

\maketitle %% required

\section{Introducción}

\section{Marco Teórico}

\subsection{Transformada de Fourier}

La transformada de Fourier descompone una función de tiempo, \textit{una señal}, en las frecuencias que lo componen,
de una manera similar a cómo un acorde musical se puede expresar como las frecuencias (o tonos) de sus notas
constituyentes. \\

La transformada de Fourier de una función del tiempo es en sí misma una función de frecuencia
de valor complejo, cuyo valor absoluto representa la cantidad de esa frecuencia presente en la función
original, y cuyo argumento complejo es el desfase de la sinusoide básica en esa frecuencia. \\

\textit{La transformada de Fourier es la representación en el dominio de la frecuencia de la señal original.} \\

El término transformada de Fourier se refiere tanto a la \textit{representación en el dominio de la frecuencia} como
a la \textit{operación matemática que asocia la representación en el dominio de la frecuencia a una función del tiempo}. \\

La transformada de Fourier no se limita a las funciones del tiempo, pero para tener un lenguaje unificado,
el dominio de la función original se conoce comúnmente como el dominio del tiempo. Por ejemplo, en el
procesamiento de imágenes, la noción de un dominio del tiempo se reemplaza por la de un dominio
espacial donde la intensidad de una señal se identifica por su posición espacial en lugar de hacerlo
en cualquier momento. \\

Para muchas funciones de interés práctico, se puede definir una operación 
que invierta el dominio de la función original. La \textit{transformación de Fourier inversa},
también llamada \textit{síntesis de Fourier}, de una representación en el dominio de la frecuencia
combina las contribuciones de todas las diferentes frecuencias para recuperar la función original del tiempo.

\subsubsection{Definiciones}

La \textbf{transformada de Fourier} de una función $f$ se denota tradicionalmente agregando un circunflejo, $\hat{f}$. Existen varias
convenciones comunes para definir la transformada de Fourier de una función integrable $f : \mathbb{R} \rightarrow \mathbb{C}$.
En el presente informe utilizaremos la siguiente definición.

\[
    \hat{f}(\xi) = \int_{- \infty}^{+ \infty} f(x) e^{-2 \pi ix \xi} dx
\]

, para cualquier $\xi \in \mathbb{R}$. \\

La razón de la convención de signos negativos en el exponente es que, e.g., en \textit{ingeniería eléctrica}, 
es común usar $f(x) = e^{2 \pi i \xi_{0} x}$ para representar una señal con cero fase inicial y 
frecuencia $\xi_{0}$ que puede ser positivo o negativo. \\

La convención del signo negativo causa el producto $e^{2 \pi i \xi_{0} x} e^{-2 \pi i \xi x}$ para 
ser $1$, \textit{frecuencia cero}, cuando $\xi = \xi_{0}$ haciendo que la integral se deverger. 
El resultado es una función delta de Dirac en $\xi = \xi_{0}$,  exactamente lo que queremos ya que esta es 
la única componente de frecuencia de la señal sinusoidal $e^{2 \pi ix \xi_{0}}$. \\

Cuando la variable independiente $x$ representa el tiempo, la variable de transformación $\xi$ representa la frecuencia
(por ejemplo, si el tiempo se mide en segundos, entonces la frecuencia está en hercios). En condiciones adecuadas, 
$f$ se determina mediante $\hat{f}$ a través de la \textbf{transformada inversa}:

\[
    f(x) = \int_{- \infty}^{+ \infty} \hat{f}(\xi) e^{2 \pi ix \xi} d \xi
\]

, para cualquier $x \in \mathbb{R}$.

\subsection{Transformada Rápida de Fourier}

Una \textbf{Transformada rápida de Fourier}, conocida por la abreviatura \textbf{FFT} (del inglés \textbf{Fast Fourier Transform}),
es un algoritmo que calcula la \textbf{transformada discreta de Fourier}, \textit{DFT}, de una secuencia, o su inversa \textit{IDFT}.
El análisis de Fourier convierte una señal de su dominio original (a menudo tiempo o espacio) a una representación
en el dominio de la frecuencia y viceversa. La DFT se obtiene descomponiendo una secuencia de valores en componentes
de diferentes frecuencias. Esta operación es útil en muchos campos, pero calcularla directamente desde la definición
es a menudo demasiado lento para ser práctico. Una FFT calcula rápidamente tales transformaciones al factorizar
la matriz DFT en un producto de factores dispersos (en su mayoría cero). Como resultado, logra reducir la complejidad
de calcular la DFT desde $\mathcal{O}(n^{2})$, que surge si uno simplemente aplica la definición de DFT,
a $\mathcal{O}(n \log{} n)$, donde $n$ es el tamaño de los datos.

\subsubsection{Definiciones}

Sean $x_{0}, \ldots, x_{N-1}$ números complejos y $k = 0, \ldots, N-1$. La DTF está definida por la formula:

\[
    X_{k} = \sum_{n = 0}^{N-1} x_{n} e^{-i2 \pi kn / N} = \sum_{n = 0}^{N-1} x_{n} w^{-kn}
\]

, donde $w = i2 \pi / N$ es la primer N-esima raiz compleja de la unidad. \\

Evaluar directamente esta definición requiere $\mathcal{O}(n^{2})$ operaciones donde hay $N$ salidas $X_{k}$, y cada
salida requiere una suma de $N$ términos. Luego, una FFT es cualquier método para calcular los mismos resultados
en las operaciones $\mathcal{O}(n \log{} n)$. Todos los algoritmos FFT conocidos requieren operaciones $\mathcal{\Theta}(n \log{} n)$,
aunque no hay pruebas conocidas de que una puntuación de complejidad menor sea imposible.

\section{Objetivos y Problemas}

En el presente trabajo se busca desarrollar y comparar algoritmos paralelos para procesadores multicore que computen
la \textbf{transformada de Fourier}, usando PThreads y OpenMP, con el objetivo de desarrollar una librería de computo
paralelo que ofrezca distintos prototipos e implementaciones para cada uno de los algoritmos elegidos. \\

Ya definido el objetivo se prosigue por definir las definiones a utilizar para la implementación de la
\textit{transformada rápida de Fourier (FFT)}. Para esto resulta interesante comparar la misma con la
definición de la \textit{transformada discreta de Fourier}.

\end{document}
